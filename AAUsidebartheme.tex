\documentclass[10pt]{beamer}
\usetheme[left]{AAUsidebar}

% If you want to change the colors of the various elements in the theme, edit and uncomment the following lines
% Change the bar and sidebar colors:
\setbeamercolor{AAUsidebar}{fg=red!20,bg=red}
%\setbeamercolor{sidebar}{bg=red!20}
% Change the color of the structural elements:
\setbeamercolor{structure}{fg=red}
% Change the frame title text color:
%\setbeamercolor{frametitle}{fg=blue}
% Change the normal text color background:
\setbeamercolor{normal text}{bg=gray!10}
% ... and you can of course change a lot more - see the beamer user manual.


\usepackage[utf8]{inputenc}
\usepackage[english]{babel}
\usepackage[T1]{fontenc}
% Or whatever. Note that the encoding and the font should match. If T1
% does not look nice, try deleting the line with the fontenc.
\usepackage{helvet}

% colored hyperlinks
\newcommand{\chref}[2]{%
  \href{#1}{{\usebeamercolor[bg]{AAUsidebar}#2}}%
}

\title[Cache y Referencias Cruzadas]% optional, use only with long paper titles
{Cache y Referencias Cruzadas} % lo que se ve en la presentacion
\subtitle{Capitulo 8, 9}  % could also be a conference name
%\subtitle{v.\ 1.4.0}  % could also be a conference name

\date{%\today
	}

\author[Wilmar Diaz Rodriguez] % optional, use only with lots of authors
{
  Wilmar Diaz Rodriguez \\    %lo que se ve en la plantilla 
  \href{mailto:wilmardiazr@gmail.com}{{\tt wilmardiazr@gmail.com}}
}
% - Give the names in the same order as they appear in the paper.
% - Use the \inst{?} command only if the authors have different
%   affiliation. See the beamer manual for an example

\institute[
%  {\includegraphics[scale=0.2]{aau_segl}}\\ %insert a company, department or university logo
  Universidad Distrital Francisco José de Caldas\\
  Maestría en CIC\\
  Tendencias en IS\\
  Bogotá DC, 16 de Abril
] % optional - is placed in the bottom of the sidebar on every slide
{% is placed on the title page
  Universidad Distrital Francisco José de Caldas\\
  Maestría en Ciencias de la información y las comunicaciones\\
  Tendencias en Ingeniería de Software\\
  Bogotá DC, 16 de Abril 2016
  
  %there must be an empty line above this line - otherwise some unwanted space is added between the university and the country (I do not know why;( )
}


% specify a logo on the titlepage (you can specify additional logos an include them in 
% institute command below
\pgfdeclareimage[height=3.5cm]{titlepagelogo}{AAUgraphics/escudoudblancoynegro} % placed on the title page
\pgfdeclareimage[height=3.5cm]{titlepagelogo2}{AAUgraphics/logoMCIC} % placed
%\pgfdeclareimage[height=1.5cm]{titlepagelogo2}{graphics/aau_logo_new} % placed on the title page
\titlegraphic{% is placed on the bottom of the title page
  \pgfuseimage{titlepagelogo}
 \hspace{1cm}\pgfuseimage{titlepagelogo2}
%  \hspace{1cm}\pgfuseimage{titlepagelogo2}
}


\begin{document}
% the titlepage
{\aauwavesbg%
\begin{frame}[plain,noframenumbering] % the plain option removes the sidebar and header from the title page
  \titlepage
\end{frame}}
%%%%%%%%%%%%%%%%

% TOC
\begin{frame}{Agenda}{}
\tableofcontents
\end{frame}
%%%%%%%%%%%%%%%%

\section{Cache}
\begin{frame}{Cache}{}
	La presente sección trata acerca del capítulo 8 del libro "\alert{Dynamic Documents with R and knitr}" 
	\begin{itemize}
		\item<1-> ¿Que entiende usted por cache?
		\item<2-> Ventajas y Desventajas
		\item<3-> Como Implemntarlo
		\item<4-> ¿Que ocurre cuando se escribe el Cache?
	\end{itemize}
\end{frame}


\subsection{¿Que entiende usted por cache?}

\begin{frame}{Cache}{¿Que entiende usted por cache?}
	\begin{block}{Uso}
		Generalmente  todo tipo de aplicaciones usan cahe por ejemplo las bases de datos, un visor de mapas en linea, etc.
	\end{block}
\end{frame}



\subsection{Ventajas y Desventajas}

\begin{frame}{Cache}{Ventajas y Desventajas}
	\begin{block}{ventajas}
		\begin{enumerate}
			\item {\tt tiempos de ejecucion}
			\item {\tt reprocesamiento}
			\item {\tt otras}
		\end{enumerate}
	\end{block}
	\begin{block}{desventajas}
		\begin{enumerate}
			\item {\tt errores en los resultados esperados}
			\item {\tt requiere incluir algunos tipos de referencias para el calculo correcto del hash}
			\item {\tt otras}
		\end{enumerate}
	\end{block}
		
\end{frame}

\subsection{Como Implemntarlo}
\begin{frame}{Cache}{Como Implemntarlo}
	\begin{block}{Como Implemntarlo}
		para implementar el cache en nuestros chunk de codigo podemos realizando cambiando las opciones en nuestro archivo rnw por medio de la opción cache=TRUE, cache=FALSE
		
	\end{block}
\end{frame}


%%%%%%%%%%%%%%%%
\section{Referencias Cruzadas}
\begin{frame}{Referencias Cruzadas}{}
	La presente sección trata acerca del capítulo 9 del libro "\alert{Dynamic Documents with R and knitr}" 
	\begin{itemize}
		\item<1-> Codigo embebido dentro de retazos de codigo
		\item<2-> Cogigos Externos
		\item<3-> Rehuso de Codigo
		\item<4-> 
	\end{itemize}
\end{frame}


%%%%%%%%%%%%%%%%

\section{Creditos }
\subsection{Bibliografia}

	

% known problems
\begin{frame}{Creditos}{Bibliografia}
	\begin{block}{Bibliografia}
		\begin{enumerate}
			\item {Libro: "Dynamic Documentswith R and knitr"
				 \\Second Edition
				 \\Autor:Yihui Xie
				 }
		\end{enumerate}
	\end{block}
\end{frame}

%%%%%%%%%%%%%%%%

\subsection{Aportes}
% help me iron out the bugs or give me some comment and suggestions
\begin{frame}{Creditos}{Aportes}
	\begin{block}{Aportes}
		\begin{enumerate}
			\item {\tt Raul Alejandro Buitrago Castellanos Plantilla}
			\item {\tt Leonel Muñoz Cedano Modificaciones cargue de Logos}
			\item {\tt Jose Nelson Pérez Castillos Libros documentos inicales }
		\end{enumerate}
	\end{block}
\end{frame}
%%%%%%%%%%%%%%%%

\subsection{Información de Contacto}
% contact information
\begin{frame}{Creditos}{Información de Contacto}
En caso de que tenga comentarios sugerencias o crea que se requiera algún tipo de correcciones por favor no dude en contactarme, si se requiere el código usado o quiere descargar los ejemplos puede conectarse al repositorio de github, a los datos que aparecen al final.
  \begin{center}
    \insertauthor\\
	Repositorio de presentacion y ejemplos\\
    \chref{https://github.com/djlocus/tendencias.git}{https://github.com/djlocus/tendencias.git}
    
  \end{center}
\end{frame}
%%%%%%%%%%%%%%%%

{\aauwavesbg
\begin{frame}[plain,noframenumbering]
  \finalpage{Gracias por la Atención Prestada!}
\end{frame}}
%%%%%%%%%%%%%%%%

\end{document}
